\section{Suchprobleme}

Suchprobleme beinhalten einen Agenten, einen Anfangszustand und einen Zielzustand. Der Agent gibt eine Lösung zurück, wie man vom ersteren zum letzteren gelangt. Ein in diesem Sinne typisches Suchproblem ist es den Weg durch ein Labyrinth zu finden.

\subsection{Fachbegriffe}
 
\begin{description}
 \item [Agent] Eine Entität, die ihre Umgebung wahrnimmt und auf diese Umgebung einwirkt. In einer Navigations-App beispielsweise das Auto.

 \item [Zustand ($s$)] Eine Konfiguration des Agenten in seiner Umgebung. Etwa die Position in einem Labyrinth

 \item [Ausgangszustand ($s_0$)] Der Zustand, von dem aus der Suchalgorithmus startet. 

 \item [Aktionen] Erlaubte Entscheidungen, die in einem Zustand getroffen werden können. 
    
    \begin{equation}
     A = f(s)
    \end{equation}
    
    In einem Labyrinth also das Bewegen in jede Richtung, die nicht von einer Wand blockiert wird.
    
 \item [Übergangsmodell $T$] Eine Beschreibung, welcher Zustand sich aus der Durchführung einer anwendbaren Aktion in einem beliebigen Zustand ergibt

    \begin{equation}
     s = f(s, a) \text{~wobei~} a \in A
    \end{equation}

 \item [Zustandsraum] Die Menge aller Zustände, die vom Anfangszustand aus durch eine beliebige Abfolge von Aktionen erreichbar sind
 
 \item [Zieltest] Die Bedingung, die bestimmt, ob ein bestimmter Zustand ein Zielzustand ist.\\
 Wenn ja: Problem gelöst. \\
 Wenn nicht: weiter suchen

 \item [Pfadkosten] Einem gegebenen Pfad zugeordnete numerische Kosten

 \item [Lösung] Eine Abfolge von Aktionen, die vom Anfangszustand zu einem Zielzustand führt
 \item [Optimale Lösung] Die Lösung mit den niedrigsten Pfadkosten unter allen Lösungen
 \item [Knoten] Datenstruktur mit folgenden Daten:
      \begin{itemize}
      \item Vorgänger, vorherige Knoten
      \item Zustand
      \item Aktion, die zu diesem Zustand geführt hat
      \item Pfadkosten vom Anfangszustand zu diesem Knoten
      \end{itemize}

      Aufgrund der Pfadkosten kann aus allen möglichen Lösungen die \enquote{günstigste} ausgewählt werden. Auch ist der Weg zum Anfangsknoten immer nachvollziehbar.

 \item [Grenzraum] Der Grenzraum enthält zunächst nur den Anfangszustand. Hinzu kommen alle Aktionen die von diesem Zustand aus erreichbar sind und noch nicht darauf hin untersucht wurden, ob sie eine Lösung darstellen. Ist der Grenzraum leer, gibt es keine Lösung. Algorithmisch
 \begin{lstlisting}
  Wiederhole:
   Wenn der Grenzraum leer ist:
      Halt! (Es gibt keine Lösung)
   
   Entferne einen Knoten aus dem Grenzraum
   
   Wenn dieser Knoten die Lösung enthält:
      Gib diesen Knoten zurück
   Sonst:
      Füge alle Knoten, die von diesem Knoten aus erreichbar sind dem Grenzraum hinzu
      Füge den diesen Knoten den untersuchten Knoten hinzu
 \end{lstlisting}

\end{description}

Es gibt zwei Wege um den Grenzraum zu verwalten:

\subsection{Als Stapel (Tiefensuche, Deep-First-Search)}

Wird der Grenzraum als Stapel (Stack) verstanden, handelt es sich um eine Tiefensuche. Diese erschöpft jede Richtung, bevor eine andere Richtung versucht wird (last-in first-out, LIFO)

\textbf{Vorteile}
\begin{itemize}
 \item Kann zufälligerweise der schnellste Algorithmus sein
\end{itemize}

\textbf{Nachteile}
\begin{itemize}
 \item Kann zufälligerweise der langstamste Algorithmus sein
 \item Es ist möglich, dass die gefundene Lösung nicht optimal ist
\end{itemize}
% \lstset{language=Python}
% \begin{lstlisting}
% #Removes an element from the stack and returns it.
% class Stack():
%    ...
%    
%    def remove(self):
%       # Exception if stack is empty
%       if self.empty():
%          raise Exception("Stack is empty")
%       else:
%          # Save the last (= newest) element
%          element = self.elements[-1]
%          # remove the last element
%          self.elements = self.elements[:-1]
%          return element
% \end{lstlisting}

\subsection{Als Warteschlange (Breitensuche, Breadth-First-Search)}

Wird der Grenzraum als Warteschlange (Queue) verstanden, handelt es sich um eine Breitensuche. Diese macht jeweils einen Schritt in jede mögliche Richtung, bevor es den nächsten Schritt in eine bestimmte Richtung macht (first-in first-out, FIFO)

\textbf{Vorteile}
\begin{itemize}
 \item Dieser Algorithmus findet garantiert die optimale Lösung
\end{itemize}

\textbf{Nachteile}
\begin{itemize}
 \item Dieser Algorithmus dauert fast garantiert länger als die minimale Ausführungszeit
 \item Im schlimmsten Fall benötigt dieser Algorithmus die längstmögliche Laufzeit.
\end{itemize}

% \lstset{language=Python}
% \begin{lstlisting}
% #Removes an element from the stack and returns it.
% class Queue():
%    ...
%    
%    def remove(self):
%       # Exception if stack is empty
%       if self.empty():
%          raise Exception("Stack is empty")
%       else:
%          # Save the first (= oldest) element
%          element = self.elements[0]
%          # remove the oldest element
%          self.elements = self.elements[1:]
%          return element
% \end{lstlisting}

\subsection{Optimierte Suche: Gierige Beste-Zuerst-Suche (Greedy Best-First Search)}
Breiten- und Tiefensuche sind beides uninformierte Suchalgorithmen. Das heißt, diese Algorithmen verwenden kein Wissen über das Problem, das sie nicht durch ihre eigene Untersuchung erworben haben. 

Die gierige Best-First-Suche schätzt aufgrund heuristische Funktion $h(n)$ welcher Knoten als nächstes untersucht werden soll. Die Effizienz des Greedy-Best-First-Algorithmus hängt davon ab, wie gut die heuristische Funktion ist. In einem Labyrinth könnte diese heuristische Funktion die Manhatten-Distanz zum Ziel sein ($|x - x_z| + |y - y_z|$)

\subsection{A* Suche}
Als Weiterentwicklung des Greedy-Best-First-Algorithmus berücksichtigt die A*-Suche nicht nur $h(n)$, die geschätzten Kosten vom aktuellen Standort zum Ziel, sondern auch $g(n)$, die bis zum aktuellen Standort angefallenen Kosten, also die Gesamtkosten ($c(n) = g(n) + h(n)$. Sobald $c(n) > c(n_{n-1})$ verlässt der Algorithmus den aktuellen Pfad kehrt zur vorherigen Option zurück, wodurch er sich selbst daran hindert einen langen, ineffizienten Weg gehen, den $h(n)$ fälschlicherweise als besten markiert hat.

Damit die A*-Suche optimal ist, muss für die heuristische Funktion h(n) gelten:
\begin{itemize}
 \item sie ist zulässig und überschätzt niemals die wahren Kosten
 \item sie ist konsistent, also muss jeden Knoten $n$ und dessen Folgeknoten $n'$ gelten $h(n) + g(n) \leq h(n') + g(n')$

\end{itemize}

\subsection{Gegnerische Suche}


\subsubsection{Minimax}
Minimax ist eine Art Algorithmus in der gegnerischen Suche und stellt Gewinnbedingungen als (-1) für eine Seite und (+1) für die andere Seite dar. Die minimierende Seite versucht, die niedrigste Punktzahl zu erzielen, und die maximierende Seite versucht, die höchste Punktzahl zu erzielen.

\begin{description}
 \item [$s_0$] Anfangszustand (in unserem Fall ein leeres 3X3-Board)
 \item [Spieler] eine Funktion, die bei einem Zustand s zurückgibt, welcher Spieler an der Reihe ist
 \item [Aktionen] eine Funktion, die bei gegebenem Zustand $s$ alle legalen Züge in diesem Zustand zurückgibt
 \item [Ergebnis] eine Funktion, die bei einem Zustand $s$ und einer Aktion $a$ einen neuen Zustand zurückgibt. Dies ist das Brett, das aus der Ausführung der Aktion $a$ im Zustand $s$ (einen Zug im Spiel machen) resultiert.
 \item[Terminal] eine Funktion, die bei einem Zustand $s$ überprüft, ob dies der letzte Schritt im Spiel ist, d.h. ob jemand gewonnen hat oder es ein Unentschieden gibt. Gibt True zurück, wenn das Spiel beendet ist, andernfalls False.
 \item[Utility(s)] eine Funktion, die bei einem Endzustand $s$ den Utility-Wert des Zustands zurückgibt: -1, 0 oder 1.
\end{description}



Rekursiv simuliert der Algorithmus alle möglichen Spiele, die ab dem aktuellen Zustand stattfinden können, bis ein Endzustand erreicht ist. Jeder Endzustand wird entweder als (-1), 0 oder (+1) bewertet.
